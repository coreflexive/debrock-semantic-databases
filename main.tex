% This file is based on a template from Stan Warford's homepage.
% Stan Warford
% Pepperdine University
% File: 221-Equations.tex
% The equation handout for Gries & Schneider, A Logical Approach to Discrete Math
% with modifications

\documentclass{article}

\usepackage{amsmath,amssymb}
\usepackage{enumerate}

\newcommand{\lgap}{2pt}                             % Line gap
\newcommand{\llgap}{6pt}                            % Larger line gap
\newcommand{\lllgap}{12pt}                          % Larger yet line gap
\newcommand{\equivs}{\ensuremath{\;\equiv\;}}       % Equivales with space
\newcommand{\equivss}{\ensuremath{\;\;\equiv\;\;}}  % Equivales with double space
\newcommand{\nequiv}{\ensuremath{\not\equiv}}       % Inequivalent
\newcommand{\impl}{\ensuremath{\Rightarrow}}        % Implies
\newcommand{\impls}{\ensuremath{\;\Rightarrow\;}}   % Implies with space
\newcommand{\nimpl}{\ensuremath{\not\Rightarrow}}   % Does not imply
\newcommand{\foll}{\ensuremath{\Leftarrow}}         % Follows from
\newcommand{\nfoll}{\ensuremath{\not\Leftarrow}}    % Does not follow from
\newcommand{\lbag}{\lbrace\hspace{-.2em}|\;}        % Left bag delimiter
\newcommand{\rbag}{\;|\hspace{-.2em}\rbrace}        % Right bag delimiter

% These macros are used for quantifications. Thanks to David Gries for sharing
\newcommand{\thedr}{\rule[-.25ex]{.32mm}{1.75ex}}   % Symbol that separates dummy from range in quantification
\newcommand{\dr}{\;\,\thedr\,\;}                    % Symbol that separates dummy from range, with spacing
\newcommand{\rb}{:}                                 % Symbol that separates range from body in quantification
\newcommand{\drrb}{\;\thedr\,{:}\;}                 % Symbol that separates dummy from body when range is missing
\newcommand{\all}{\forall}                          % Universal quantification
\newcommand{\ext}{\exists}                          % Existential quantification
\newcommand{\guard}{[\negthinspace ]}               % Rectangle for guard

% Macros for proof hints
\newcommand{\Gll} {\langle}                         % Open hint
\newcommand{\Ggg} {\rangle}                         % Close hint
\newlength{\Glllength}                              % Length of open hint symbol
\settowidth{\Glllength}{$.\Gll$}
\newcommand{\Hint}[1]     {\ \ \ $\Gll              \mbox{#1} \Ggg$ }   % Single line hint
\newcommand{\Hintfirst}[1]{\ \ \ $\Gll              \mbox{#1}$ }        % First line of multiline hint
\newcommand{\Hintmid}[1]  {\ \ $\hspace{\Glllength} \mbox{#1}$ }        % Middle line of multiline hint
\newcommand{\Hintlast}[1] {\ \ $\hspace{\Glllength} \mbox{#1} \Ggg$ }   % Last line of multiline hint

% Single and double quotes
\newcommand{\Lq}{\mbox{`}}
\newcommand{\Rq}{\mbox{'}}
\newcommand{\Lqq}{\mbox{``}}
\newcommand{\Rqq}{\mbox{''}}

% Readable combinators
\newcommand{\within}{\subseteq}
\newcommand{\contains}{\supseteq}
\newcommand{\union}{\cup}
\newcommand{\intersection}{\cap}
\newcommand{\minus}{-}
\newcommand{\powerset}{\mathcal{P}}
\newcommand{\Union}{\bigcup}
\newcommand{\converse}{^\circ}

\DeclareMathOperator{\dom}{dom}
\DeclareMathOperator{\rng}{rng}

\DeclareMathOperator{\simple}{simple}

\newcommand{\cardinality}[1]{|#1|}

\begin{document}
\title{Theorems from De Brock's Foundations of Semantic Databases}
\author{Adam W. Grant}
\date{\today}

\maketitle

\section{Basic Mathematical Concepts}

\subsection{Sets}
\paragraph{Definition 1:}
$\mathbb{N} = \{ k \in \mathbb{Z} : k \geq 0 \}$

\paragraph{Definition 2:}
If $A$ and $B$ are sets, then:
\begin{enumerate}[(a)]
    \item $A \within B  \equivs \forall( x : x \in A : x \in B)$
    \item $B \contains A  \equivs A \within B$
    \item $A \subset B \equivs A \within B \land A \neq B$
    \item $A \union B  =  \{ x : x \in A \lor x \in B \}$
    \item $A \intersection B  =  \{ x : x \in A \land x \in B \}$
    \item $A \minus B  =  \{ x : x \in A \land x \notin B \}$
    \item $A \times B  =  \{ x,y : x \in A \land x \in B : (x,y) \}$
    \item $\powerset.A  =  \{ X : X \within A \}$
\end{enumerate}

\paragraph{Definition 3:} If $m \in \mathbb{Z}$ and $n \in \mathbb{Z}$, then:
\begin{enumerate}[(a)]
    \item $[m .. n] = \{ k \in \mathbb{Z} : m \leq k \land k \leq n \}$
    \item $[m .. n) = \{ k \in \mathbb{Z} : m \leq k \land k < n \}$
    \item $[m .. ) = \{ k \in \mathbb{Z} : m \leq k \}$
\end{enumerate}

\paragraph{Definition 4:} If $n \in \mathbb{N} - \{0\}$
\begin{enumerate}[(a)]
    \item $Snn.n = [10^{n-1}..10^n-1]$
    \item $Int.n = [-(10^{n-1})..10^n-1]$
\end{enumerate}

\paragraph{Definition 5:} If W is a set of sets, then:\[\bigcup W  =  \{ x : \exists(A : A \in W : x \in A) \}\]

\paragraph{Lemma 1:}  If $A$ and $B$ are sets, then:
\begin{enumerate}[(a)]
    \item $\Union \emptyset = \emptyset$
    \item $\Union \{A\} = A$
    \item $\Union \{A, B\} = A \union B$
\end{enumerate}

\paragraph{Definition 6:}  If $R$ is a set of ordered pairs, then:
\begin{enumerate}[(a)]
    \item $\dom.R = \{ x,y : (x,y) \in R : x \}$
    \item $\rng.R = \{ x,y : (x,y) \in R : y \}$
    \item $R \converse = \{ x,y : (x,y) \in R : (y,x) \}$
\end{enumerate}

\paragraph{Lemma 2:} If $R$ and $P$ are sets of ordered pairs and $D \within R$, then:
\begin{enumerate}[(a)]
    \item $R \union P$, $R \intersection P$, $R \minus P$, $D$, and $R \converse$ are also sets of ordered pairs.
    \item \begin{enumerate}[i.]
        \item $\dom(R \union P) = \dom.R \union \dom.P$
        \item $\rng(R \union P) = \rng.R \union \rng.P$
    \end{enumerate}
    \item \begin{enumerate}[i.]
        \item $\dom(R \intersection P) \within \dom.R \intersection \dom.P$
        \item $\rng(R \intersection P) \within \rng.R \intersection \rng.P$
    \end{enumerate}
    \item \begin{enumerate}[i.]
        \item $\dom(R \minus P) \contains \dom.R \minus \dom.P$
        \item $\rng(R \minus P) \contains \rng.R \minus \rng.P$
    \end{enumerate}
    \item \begin{enumerate}[i.]
        \item $\dom.D \within \dom.R$
        \item $\rng.D \within \rng.R$
    \end{enumerate}
    \item \begin{enumerate}[i.]
        \item $\dom(R \converse) = \rng.R$
        \item $\rng(R \converse) = \dom.R$
    \end{enumerate}
    \item \begin{enumerate}[i.]
        \item $(R \union P)\converse = R\converse \union P\converse$
        \item $(R \intersection P)\converse = R\converse \intersection P\converse$
        \item $(R \minus P)\converse = R\converse \minus P\converse$
        \item $(R \converse)\converse = R$
    \end{enumerate}
\end{enumerate}


\subsection{Simple Relations}
What De Brock calls a function is not a function at all: it is merely a simple (a.k.a deterministic) relation.  That convention is clumsy and I do not subscribe to it.  Instead I will use the naming conventions of Oliveira (after Freyd and Scedrov).

\paragraph{Definition 7:} For $R$, a set of ordered pairs:\[ \simple.R  \equivs  \forall(x,y : (x,y) \in R : \forall(x',y' : (x',y') \in R : x = x' \impl y = y')) \]

\paragraph{Lemma 3:} If $R$ and $S$ are simple relations, and $R \within S$, then:
\begin{enumerate}[(a)]
    \item $\simple.R$
    \item \begin{enumerate}[i.]
        \item $\simple(R \intersection S)$
        \item $\simple(R \minus S)$
    \end{enumerate}
    \item $\cardinality{\rng.S} \leq \cardinality{\dom.S}$
    \item \begin{enumerate}[i.]
        \item $\simple.\emptyset$
        \item $\dom.\emptyset = \rng.\emptyset = \emptyset$
    \end{enumerate}
\end{enumerate}

\end{document}
